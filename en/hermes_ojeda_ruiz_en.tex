\documentclass[11pt,letterpaper,sans]{moderncv}

\moderncvstyle{casual}
\moderncvcolor{orange}

\usepackage[utf8]{inputenc}

\usepackage[scale=0.75]{geometry}
\setlength{\hintscolumnwidth}{2.5cm}

\firstname{Hermes}
\familyname{Ojeda Ruiz}
\address{Segunda Privada de la Noria 209-A. Centro.}{Oaxaca de Juarez, Oaxaca  C.P. 68000}
\phone{(+52)951~513~0827}
\mobile{951~192~5750}
\email{hermes.ojeda@logicalbricks.com}
%\photo[67pt][0.4pt]{picture.png}                  % '64pt' is the height the picture must be resized to, 0.4pt is the thickness of the frame around it (put it to 0pt for no frame) and 'picture' is the name of the picture file; optional, remove the line if not wanted

\begin{document}
\maketitle

\section{General Development Languages and Tools}
\cvcomputer{Languages}{Ruby (advanced), C (advanced), C++ (intermediate), Java (basic), PHP (basic), Python (basic), SQL (intermediate)}{Frameworks and Specifications}{Ruby on Rails, CakePHP, Qt, Joomla, JQuery, WebServices, RESTful}
\cvcomputer{Web}{HTML (advanced), JavaScript (intermediate), CSS (intermediate)}{IDE}{Vim (advanced), Eclipse, Netbeans, Aptana}
\cvcomputer{DBMS}{MySQL, PostgreSQL, Sqlite}{Operating System}{Linux(Debian, RedHat, Ubuntu, ArchLinux, Chakra), *nix (OpenBSD)}
\cvcomputer{System Administration}{Kernel-based Virtual Machines(KVM), Linux Terminal Server Project(LTSP), RedHat Enterprise Virtualization, Apache, Packet Filter(PF), Samba, Shorewall, Squid, DHCP, BIND, Tomcat}{SCM}{Git, Subversion}

\section{RoR-related Development Tools}
\cvcomputer{Testing}{Rspec, Cucumber, Capybara, Shoulda, FactoryGirl, Minitest}{Deployment}{ Capistrano, Unicorn, Puma, Nginx, Apache, Passenger, REE}
\cvcomputer{Views}{Haml, JQuery, Coffeescript, Saas, Bootstrap}{Others}{RVM, Rbenv, JRuby, Sinatra, Savon, Waz-Storage, Delayed\_job, Sidekiq, Spree, Watir, Mechanize }


\section{Professional Experience}

\cventry{2011--present}{LogicalBricks}{CoFounder, Scrum Master and Developer}{Oaxaca de Juárez, Oaxaca}{}{ Developing, maintaining and upgrading the company's main projects. Implementation of agile practices including Scrum, Pair Programming, TDD and BDD.\newline{}%
  \newline{}
  Example projects:%
  \begin{itemize}%
    \item Modeling and coding a system for a PAC (Certified Digital Invoices Provider) applicant company to seal and stamp digital invoices implementing the model given by the Mexican Tax Agency [developed with RoR during 3 weeks]
    \item Planning, modeling and coding a system for financial institutions to report an specific tax to the Mexican Tax Agency [developed with RoR during 1.5 months, using BDD]
    \item Planning, modeling and coding a billing system with two-dimensional barcodes, following the specification given by the Mexican Tax Agency [developed with RoR during 2 months, using BDD]
    \item Planning, modeling and coding a web application to download automatically the digital invoices from the Mexican Tax Agency, because this process is manual and very tedious, and the download process was accomplished with a little complex web-scrapping. [developed with RoR during 1 month, using BDD]
    \item Gem development: fm\_timbrado\_cfdi to generate simple connections to "Facturación Moderna" a PAC certified by SAT, with OpenSource license.
    \item Developing "Infacto" a service to generate digital invoices and payroll valid for Mexican Tax Agency.
    \item Building a Learning Management System integrated with an e-commerce application for a Canadian company [Remotely outsourced]
    \item Web application development to manage budget, expenses and bills for a State Goverment Office in charge of Public Service Announcement.
    \item OpenSource projects contributions: waz-storage, github-notifier, Nela, Chakra GNU/Linux.
\end{itemize}}

\cventry{2010--2011}{Freelance}{Consultant/SysAdmin/Developer}{}{}{ Featured actitivies:%
  \begin{itemize}%
    \item Setting up a Debian server with DHCP, DNS, Squid and Shorewall to \href{http://apointmexico.com/}{Apoint México}, an important mexican company.
    \item Planning the OpenSource-based services for \href{http://highmicro.com}{HighMicro}.
    \item Setting up and manage an OpenBSD firewall on embedded hardware with routing, load balancing and equally-share bandwidth on a Bussines Center in Mexico City.
    \item Setting up and managing a Red Hat Enterprise Virtualization solution to deploy Virtual Desktops with load balancing.
    \item Course: Introduction to Ruby on Rails to \href{http://apointmexico.com/}{Apoint México}.
    \item Customization of the \href{https://github.com/hermes-logicalbricks/Efemerides}{Efemérides} Joomla! extension, for \href{http://deadcanadians.ca/}{DeadCanadians} webpage.
  \end{itemize}}

\cventry{2008--2010}{KadaSoftware, Universidad Tecnológica de la Mixteca}{SysAdmin/Developer}{Huajuapan de León, Oaxaca}{}{ Consolidation of an OpenSource-based business model, transfering knowledge of web-development techniques and system administration of servers and network services of the company.\newline{}%
  \newline{}
  Featured activities:%
  \begin{itemize}%
    \item Developing the software's activation system using Java and encryption. The main goal was the addition of product keys and expiration dates to the software in a secure way.
    \item Managing web, mail and development servers, and the system to control the payroll of some companies.
    \item Plannig and controlling the Web and OpenSource services of the company.
    \item Joomla! extension developer. Developing extension to show weather information for a government website, and an extension to manage a list of partners on the client's website.
    \item Course: ``Developing Joomla! components using MVC pattern'' to internal workers.
    \item Setting up servers: A File server, Debian and Ubuntu internal mirrors and a Virtualization server using KVM.
    \item Managing the network connection using an OpenBSD firewall.
    \item Introduce Agile Development technologies like Ruby on Rails into the company.
\end{itemize}}

\section{Education}
\cventry{2000--2007}{Computer Engineering}{Technological University of the Mixteca}{Huajuapan de Leon, Oaxaca}{\textit{Computer Engineering, Bachelor's degree}}{Graduated through CENEVAL, with High Academic Performance Testimony, 2008.}

\section{Achievements and awards}
\cventry{2008}{Honorable Mention}{ACM ICPC 2007 World Finals}{Tokyo, Japan}{}{}
\cventry{2007}{First place}{2007 ACM ICPC Mexico and Central America}{Puebla, México}{}{}
\cventry{2005}{Honorable Mention}{ACM ICPC 2006 World Finals}{Shanghai, China}{}{}
\cventry{2005}{Second place}{2005 ACM ICPC Mexico and Central America}{Veracruz, México}{}{}

\section{Languages}
\cvlanguage{Spanish}{Native}{Native}
\cvlanguage{English}{Proficient}{Currently taking an English for Business course}

\section{Workshops and Conferences}

\subsection{As Attendee}
\cventry{2006}{Scientific Visualitation}{III Pan-American Advanced Institute(PASI)}{Huajuapan, Oaxaca}{}{Postgraduate courses for Computer Science and Engineering.}
\cventry{2006}{Workshop 'Programación Extrema (eXtreme Programming)'}{I Simposium of Free Software on the Mixteca}{Huajuapan, Oaxaca}{}{eXtreme Programming basic introduction using Python lenguage.}
\cventry{2007}{Google ACM Event}{Google}{New York, USA}{}{Conferences and meeting about the Google environment for ACM ICPC 2007 World Finals.}
\cventry{2008}{Workshop 'Microcontroller programming with FLOSS'}{II International Congress of System Engineering}{Tehuacán, Puebla}{}{Introduction to the free software tools in microcontroller programming.}
\cventry{2013}{CSD Full-Track}{Scrum Alliance/Kleer/Scrum.org.mx}{México, DF}{}{ Course of preparation to Scrum Developer Certification.}

\subsection{As Speaker}
\cventry{2007}{Local ACM ICPC 2008}{UTM}{Huajuapan, Oaxaca}{}{Jury and problem proposer on the local ACM ICPC 2008 of the Universidad Tecnológica de la Mixteca.}
\cventry{2007}{OLPC Project}{6 Linux's Seminar}{Tehuacán, Puebla}{}{Conference with Manuel Montoya about OLPC project and its educational implications.}
\cventry{2007}{Introduction to the ACM ICPC}{6 Linux's Seminar}{Tehuacán, Puebla}{}{Conference about the rules, strategies and experiences on the ACM ICPC.}
\cventry{2007}{Workshop 'C Programming' }{6 Linux's Seminar}{Tehuacán, Puebla}{}{Workshop to movitate the attendants into the C programming using really simple code but that allow to get interesting results.}
\cventry{2007}{Workshop 'C with SDL' }{6 Linux's seminar}{Tehuacán, Puebla}{}{Introductory workshop to use the SDL library with C.}
\cventry{2008}{Develop Joomla! extensions using MVC pattern}{II International Congress of System Engineering}{Tehuacán, Puebla}{}{Introduction to Joomla! components development using the MVC design pattern.}
\cventry{2009}{Pilot Scheme 'Network of recycled computers'}{}{San Baltazar Guelavila, Oaxaca}{}{Presentation of a recycled computers network using LTSP to the telesecondary school of San Baltazar Guelavila.}
\cventry{2010}{Recycled computers network using LTSP}{Appropiate Technologies National Forum}{Oaxaca, Oaxaca}{}{Conference about the experience using recycled computers to build computer labs to needy schools.}
\cventry{2010}{Kids on Computers}{III FreeSoftware Simposium of the Mixteca}{Huajuapan, Oaxaca}{}{Conference with Thomas Peters about the work experience with Kids on Computers foundation.}
\cventry{2010}{Ruby on Rails}{III FreeSoftware Simposio of the Mixteca}{Huajuapan, Oaxaca}{}{Introduction to Ruby on Rails framework.}
\cventry{2010}{Ruby on Rails}{1er. Aniversary of Instituto Tecnológico Superior de Teposcolula}{Teposcolula, Oaxaca}{}{Introduction to Ruby on Rails framework.}
\cventry{2010}{Workshop 'Introduction to OpenBSD'}{1er. Aniversary of Instituto Tecnológico Superior de Teposcolula}{Teposcolula, Oaxaca}{}{Introductory workshop of system installation and basic nertwork-services configuration.}
\cventry{2011}{Using Free Software in Education}{Regional Forum: Development Plan of the State 2011-2016}{Oaxaca, Oaxaca}{}{Conference about using FreeSofware on Education like a support tool to improve education in Oaxaca.}
\cventry{2011}{FreeSoftware Tools to start a Software Development Company}{V FreeSofware Simposium of the Mixteca}{Huajuapan, Oaxaca}{}{Conference about FreeSoftware tools to keep working a Sofware Development Company, with cost reduction but with professional results.}
\cventry{2011}{WorkShop 'Soekris and OpenBSD'}{V FreeSoftware Simposium of the Mixteca}{Huajuapan, Oaxaca}{}{Workshop about how to use OpenBSD like firewall and how setting up directly in embedded hardware.}
\cventry{2012}{A FreeSoftware-based company - One year later}{VI FreeSoftware Simposium of the Mixteca}{Huajuapan, Oaxaca}{}{Conference about the one-year experience of LogicalBricks Solutions.}
\cventry{2013}{BDD with Rails and Cucumber}{VII FreeSoftware Simposium of the Mixteca}{Huajuapan, Oaxaca}{}{Conference about how to use Cucumber for BDD in Rails.}
\cventry{2013}{WorkShop 'BDD with Rails and Cucumber'}{VII FreeSoftware Simposium of the Mixteca}{Huajuapan, Oaxaca}{}{Workshop about how to use Cucumber for BDD in Rails.}
\cventry{2014}{Refactoring - Removing the smelliness to the code}{VIII FreeSoftware Simposium of the Mixteca}{Huajuapan, Oaxaca}{}{Conference and workshop about refactoring techniques.}
\cventry{2015}{Why am I happy developing in Ruby?}{Computer Science's Week}{Ixtlán, Oaxaca}{}{Talk about why do I love programming with Ruby.}
\cventry{2015}{Workshop Git Introduction}{Computer Science's Week}{Ixtlán, Oaxaca}{}{Workshop to learn the most used git commands and features.}
\cventry{2015}{Artoo, a framework to rule them all }{oaxaca.rb}{Oaxaca, Oaxaca}{}{Talk about Artoo using Arduino and OpenCV.}


\subsection{As Organizer}
\cventry{2011}{TechDay Oaxaca 2011}{}{Oaxaca, Oaxaca}{}{Conference focused at the government public sector in order to introduce to the IT staff the technologies from different companies like IBM, Huawei, RedHat, Team, MGS and LogicalBricks.}{}
\cventry{2013}{Global Day Code Retreat 2013}{}{Oaxaca, Oaxaca}{}{Organizing a Global Day Code Retreat in Oaxaca City}{}
\cventry{2014}{Global Day Code Retreat 2014}{}{Oaxaca, Oaxaca}{}{Organizing a Global Day Code Retreat in Oaxaca City}{}
\cventry{2014}{1st Coding Dojo}{}{Oaxaca, Oaxaca}{}{Organizing the oaxaca.rb's first Coding Dojo}{}
\cventry{2014-2015}{Coding Dojos}{}{Oaxaca, Oaxaca}{}{Organizing every two weeks the oaxaca.rb's Coding Dojos and talks}{}

\subsection{OpenSource contributions and other activities}
\cventry{2014--present}{\href{http://oaxacarb.org}{oaxaca.rb}}{Member/Organizer}{}{}{\begin{itemize}%
    \item Organizing every two weeks Coding Dojos and Talks.
    \item Organizing special talks with foreing speakers.
    \item Facilitating the coding dojos.
  \end{itemize}}

\cventry{2009--present}{\href{http://kidsoncomputers.org}{Kids on Computers}}{Voluntario}{}{}{\begin{itemize}%
    \item Setting up a computer lab with 30 recycled computers on the primary school ``Dieciocho de Marzo''. Huajuapan, Oaxaca.
    \item Setting up a computer lab with recycled computers using LTSP on a cernter for kids with disabilities ``Centro de Atención Múltiple 27''. Tlaxiaco, Oaxaca.
    \item Setting up a computer lab with recycled computers on ``Centro Santo Domingo'' school. Huajuapan, Oaxaca.
    \item Supporting all the computer labs on the south of México.
  \end{itemize}}

\cventry{2009--present}{\href{http://hermes-logicalbricks.github.com/Efemerides/}{Efemérides}}{Maintainer}{Joomla! extension}{}{\begin{itemize}%
    \item A very simple Joomla!'s extension to manage important historical events.
    \item Now with Joomla 2.5 support.
  \end{itemize}}

\cventry{2011--present}{\href{http://github.com/LogicalBricks/wicap-php}{wicap-php}}{Maintainer}{}{}{\begin{itemize}%
    \item Captive portal designed to work with OpenBSD.
    \item With support for OpenBSD 5.1
  \end{itemize}}

\cventry{2012--present}{\href{http://nelaproject.blogspot.com.es/}{Nela}}{Developer}{}{}{\begin{itemize}%
    \item Project with the goal to support learning of Braille's writing to visually impaired people.
    \item Mantainer of CCR Chakra/Linux package.
  \end{itemize}}

\cventry{2012--present}{\href{http://chakra-linux.org/}{Chakra GNU\/Linux}}{Community Packager}{}{}{Mantainer of 3 CCR packages: epson--inkjet--printer--escpr, nela and uml\_utilities}

\cventry{2011--present}{Other contributions}{Developer}{}{}{Some aditional contributions to OpenSource projects:
  \begin{description}%
    \item[\href{http://github.com/zakird/wkhtmltopdf\_binary\_gem/}{wkhtmltopdf\_binary}] Cleaning the source code to allow the gemto be installed directly from its github repository.
    \item[\href{http://github.com/johnnyhalife/waz-storage/}{waz-storage}] Implementing the feature for Shared Access Signature autentication.
    \item[\href{http://github.com/abiczo/github-notifier/}{github-notifier}] Implementing Organizations support using the Github's new API.
  \end{description}}
\end{document}
